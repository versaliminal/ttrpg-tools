\documentclass[10pt,letterpaper,portrait]{witches}
%\geometry{left=1.25cm,right=1.25cm,top=1.5cm,bottom=1.5cm,columnsep=1.2cm}

\makeindex

\begin{document}

\fancyhead{}
\renewcommand{\headrulewidth}{0pt}
\fancyfoot[C]{
    \colorbox{VLRed}{
        \makebox{
            \Huge\bfseries\color{white}\textgoth{\thepage}
        }
    }
}
\fancyfoot[L]{
    {\large \color{MidGrey} Quinn McCaffrey}
}
\fancyfoot[R]{
    {\large \color{MidGrey} VERSALIMINAL Press}
}

\bigskip
\bigskip
\begin{center}
    \includegraphics[width=\textwidth]{img/title.png}\newpage
\end{center}

\rulesHeader

\begin{center}
    {\noindent \bfseries{Unless othwerise noted, follow the rules and guidelines provided in Mythic Bastionland}}
\end{center}

\columnratio{0.5}
\setlength{\columnsep}{3em} 
\begin{paracol}{2}

\sectionName{Witches and Covens}
Witches seek not glory, but anarchic liberation, rebirth, and retribution. Drawn to mythic circumstance, they are often regarded as myths themselves and may avail themselves of secrecy or theater as needed.\\

In a sign of troubled times, you have been brought together by sign and contrivance and have joined as a coven. While individuals may separate or even die, your journey and fate is now bound to the collective.\\

\begin{center}
\bfseries{
Trespass and fear not the path of no return \\
Seek the signs and omens \\
There is no truth but in old women's tales
}
\end{center}

\sectionName{Creating a Witch}
First roll to set your \keyword{VIG}, \keyword{CLA}, \keyword{SPI}, and \keyword{GD} as in the base game. Then Roll to select or choose a witch. Each witch description provides:\\
\itemName{Property}{Any items in your possession}
\itemName{Passion}{A means to restore your spirit}
\itemName{Craft}{Abilities which are special to you}
\itemName{Initiating Omen}{The strange event that caused you to seek witchdom}
\\
\par\noindent
Every witch carries two daggers \statBlock{d6}, robust clothing \statBlock{A1}, wilderness provisions, wildrope (strong, blends in to nature), and knows the luciform craft (provides weak illumination).
\\
\par\noindent
Every witch has also learned the core feats: \keyword{Harrow} {\small(Smite)}, \keyword{Scry} {\small(Focus)}, and \keyword{Hex} {\small(Deny)} and associated \keyword{Gambits}.

\sectionName{Knights}
Knights are often key parts of the power structures of realms, and those that aren't often wish to inherit those structures. Be wary of them; although they often know little of witches, they can react quickly when a perceived threat appears. Never face a knight alone.

\switchcolumn

\sectionName{Beginings and Honors}
Witches do not seek glory, but honors are bestowed on them as titles through the completion of a myths or other notable deeds. Ex: "Lorel, theif of bound heart". An elder witch may be widely known by her titles.
\\
\par\noindent You may choose to start a s Whelp with no titles or an established Dam with one or more titles.

\sectionName{Travel and Exploration}
Witches move fluidly through tough terrain, but rarely acquire a mount. Keen explorers may encounter fairy circles which link remote areas.
\\
\par\noindent When exploring, a witch may be able to find and gain information from a willing animal or nature spirit.

\sectionName{Combat}
Witches are rarely armored and must be careful when dealing with powerful enemies.

\sectionName{Myths and Seers}
Witches appear in mythic circumstances like mushrooms after rain. Be clever, be adaptable, and be careful and you might find success.\\%

Witches may choose to seek the guidance of Seers as Knights or others would and Seers are generally no more or less willing to help or exploit them.

\sectionName{Warfare, Dominion, and Authority}
Subterfuge, influence, and instigation are generally preferred to direct rule and open warfare. Any witch that openly takes a position of power will be stripped of her titles, but this is not unprecedented.

\end{paracol}
\clearpage

%\tableofcontents

\include{rendered/1-Poesy.tex}
\include{rendered/2-Prolix.tex}
\include{rendered/3-Volta.tex}

\include{rendered/1-The First Ruler.tex}
\include{rendered/2-The Waxen Slain.tex}
\include{rendered/3-The White Fountain.tex}

%\printindex

\end{document}
